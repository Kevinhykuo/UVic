



    
\documentclass[11pt]{article}
    
    \usepackage{parskip}
    \setcounter{secnumdepth}{0} %Suppress section numbers
    \usepackage[breakable]{tcolorbox}
    \tcbset{nobeforeafter}
    \usepackage{needspace}
    \usepackage{minted}
    \usemintedstyle{jupyter_python}
    
    \usepackage[T1]{fontenc}
    % Nicer default font (+ math font) than Computer Modern for most use cases
    \usepackage{mathpazo}

    % Basic figure setup, for now with no caption control since it's done
    % automatically by Pandoc (which extracts ![](path) syntax from Markdown).
    \usepackage{graphicx}
    % We will generate all images so they have a width \maxwidth. This means
    % that they will get their normal width if they fit onto the page, but
    % are scaled down if they would overflow the margins.
    \makeatletter
    \def\maxwidth{\ifdim\Gin@nat@width>\linewidth\linewidth
    \else\Gin@nat@width\fi}
    \makeatother
    \let\Oldincludegraphics\includegraphics
    % Set max figure width to be 80% of text width, for now hardcoded.
    \renewcommand{\includegraphics}[1]{\Oldincludegraphics[width=.8\maxwidth]{#1}}
    % Ensure that by default, figures have no caption (until we provide a
    % proper Figure object with a Caption API and a way to capture that
    % in the conversion process - todo).
    \usepackage{caption}
    \DeclareCaptionLabelFormat{nolabel}{}
    \captionsetup{labelformat=nolabel}

    \usepackage{adjustbox} % Used to constrain images to a maximum size 
    \usepackage{xcolor} % Allow colors to be defined
    \usepackage{enumerate} % Needed for markdown enumerations to work
    \usepackage{geometry} % Used to adjust the document margins
    \usepackage{amsmath} % Equations
    \usepackage{amssymb} % Equations
    \usepackage{textcomp} % defines textquotesingle
    % Hack from http://tex.stackexchange.com/a/47451/13684:
    \AtBeginDocument{%
        \def\PYZsq{\textquotesingle}% Upright quotes in Pygmentized code
    }
    \usepackage{upquote} % Upright quotes for verbatim code
    \usepackage{eurosym} % defines \euro
    \usepackage[mathletters]{ucs} % Extended unicode (utf-8) support
    \usepackage[utf8x]{inputenc} % Allow utf-8 characters in the tex document
    \usepackage{fancyvrb} % verbatim replacement that allows latex
    \usepackage{grffile} % extends the file name processing of package graphics 
                         % to support a larger range 
    % The hyperref package gives us a pdf with properly built
    % internal navigation ('pdf bookmarks' for the table of contents,
    % internal cross-reference links, web links for URLs, etc.)
    \usepackage{hyperref}
    \usepackage{longtable} % longtable support required by pandoc >1.10
    \usepackage{booktabs}  % table support for pandoc > 1.12.2
    \usepackage[inline]{enumitem} % IRkernel/repr support (it uses the enumerate* environment)
    \usepackage[normalem]{ulem} % ulem is needed to support strikethroughs (\sout)
                                % normalem makes italics be italics, not underlines
    

    \let\Oldtex\TeX     % provide compatibility with nbconvert <= 5.3.1
    \let\Oldlatex\LaTeX % pre-included in nbconvert > 5.3.1 so redundant
    
    % Colors for the hyperref package
    \definecolor{urlcolor}{rgb}{0,.145,.698}
    \definecolor{linkcolor}{rgb}{.71,0.21,0.01}
    \definecolor{citecolor}{rgb}{.12,.54,.11}

    % ANSI colors
    \definecolor{ansi-black}{HTML}{3E424D}
    \definecolor{ansi-black-intense}{HTML}{282C36}
    \definecolor{ansi-red}{HTML}{E75C58}
    \definecolor{ansi-red-intense}{HTML}{B22B31}
    \definecolor{ansi-green}{HTML}{00A250}
    \definecolor{ansi-green-intense}{HTML}{007427}
    \definecolor{ansi-yellow}{HTML}{DDB62B}
    \definecolor{ansi-yellow-intense}{HTML}{B27D12}
    \definecolor{ansi-blue}{HTML}{208FFB}
    \definecolor{ansi-blue-intense}{HTML}{0065CA}
    \definecolor{ansi-magenta}{HTML}{D160C4}
    \definecolor{ansi-magenta-intense}{HTML}{A03196}
    \definecolor{ansi-cyan}{HTML}{60C6C8}
    \definecolor{ansi-cyan-intense}{HTML}{258F8F}
    \definecolor{ansi-white}{HTML}{C5C1B4}
    \definecolor{ansi-white-intense}{HTML}{A1A6B2}

    % commands and environments needed by pandoc snippets
    % extracted from the output of `pandoc -s`
    \providecommand{\tightlist}{%
      \setlength{\itemsep}{0pt}\setlength{\parskip}{0pt}}
    \DefineVerbatimEnvironment{Highlighting}{Verbatim}{commandchars=\\\{\}}
    % Add ',fontsize=\small' for more characters per line
    \newenvironment{Shaded}{}{}
    \newcommand{\KeywordTok}[1]{\textcolor[rgb]{0.00,0.44,0.13}{\textbf{{#1}}}}
    \newcommand{\DataTypeTok}[1]{\textcolor[rgb]{0.56,0.13,0.00}{{#1}}}
    \newcommand{\DecValTok}[1]{\textcolor[rgb]{0.25,0.63,0.44}{{#1}}}
    \newcommand{\BaseNTok}[1]{\textcolor[rgb]{0.25,0.63,0.44}{{#1}}}
    \newcommand{\FloatTok}[1]{\textcolor[rgb]{0.25,0.63,0.44}{{#1}}}
    \newcommand{\CharTok}[1]{\textcolor[rgb]{0.25,0.44,0.63}{{#1}}}
    \newcommand{\StringTok}[1]{\textcolor[rgb]{0.25,0.44,0.63}{{#1}}}
    \newcommand{\CommentTok}[1]{\textcolor[rgb]{0.38,0.63,0.69}{\textit{{#1}}}}
    \newcommand{\OtherTok}[1]{\textcolor[rgb]{0.00,0.44,0.13}{{#1}}}
    \newcommand{\AlertTok}[1]{\textcolor[rgb]{1.00,0.00,0.00}{\textbf{{#1}}}}
    \newcommand{\FunctionTok}[1]{\textcolor[rgb]{0.02,0.16,0.49}{{#1}}}
    \newcommand{\RegionMarkerTok}[1]{{#1}}
    \newcommand{\ErrorTok}[1]{\textcolor[rgb]{1.00,0.00,0.00}{\textbf{{#1}}}}
    \newcommand{\NormalTok}[1]{{#1}}
    
    % Additional commands for more recent versions of Pandoc
    \newcommand{\ConstantTok}[1]{\textcolor[rgb]{0.53,0.00,0.00}{{#1}}}
    \newcommand{\SpecialCharTok}[1]{\textcolor[rgb]{0.25,0.44,0.63}{{#1}}}
    \newcommand{\VerbatimStringTok}[1]{\textcolor[rgb]{0.25,0.44,0.63}{{#1}}}
    \newcommand{\SpecialStringTok}[1]{\textcolor[rgb]{0.73,0.40,0.53}{{#1}}}
    \newcommand{\ImportTok}[1]{{#1}}
    \newcommand{\DocumentationTok}[1]{\textcolor[rgb]{0.73,0.13,0.13}{\textit{{#1}}}}
    \newcommand{\AnnotationTok}[1]{\textcolor[rgb]{0.38,0.63,0.69}{\textbf{\textit{{#1}}}}}
    \newcommand{\CommentVarTok}[1]{\textcolor[rgb]{0.38,0.63,0.69}{\textbf{\textit{{#1}}}}}
    \newcommand{\VariableTok}[1]{\textcolor[rgb]{0.10,0.09,0.49}{{#1}}}
    \newcommand{\ControlFlowTok}[1]{\textcolor[rgb]{0.00,0.44,0.13}{\textbf{{#1}}}}
    \newcommand{\OperatorTok}[1]{\textcolor[rgb]{0.40,0.40,0.40}{{#1}}}
    \newcommand{\BuiltInTok}[1]{{#1}}
    \newcommand{\ExtensionTok}[1]{{#1}}
    \newcommand{\PreprocessorTok}[1]{\textcolor[rgb]{0.74,0.48,0.00}{{#1}}}
    \newcommand{\AttributeTok}[1]{\textcolor[rgb]{0.49,0.56,0.16}{{#1}}}
    \newcommand{\InformationTok}[1]{\textcolor[rgb]{0.38,0.63,0.69}{\textbf{\textit{{#1}}}}}
    \newcommand{\WarningTok}[1]{\textcolor[rgb]{0.38,0.63,0.69}{\textbf{\textit{{#1}}}}}
    
    
    % Define a nice break command that doesn't care if a line doesn't already
    % exist.
    \def\br{\hspace*{\fill} \\* }
    % Math Jax compatability definitions
    \def\gt{>}
    \def\lt{<}
    % Document parameters
    \title{Assignment 3}
    
    
    
% Pygments definitions
    
    \makeatletter
    \newcommand*\@iflatexlater{\@ifl@t@r\fmtversion}
    \@iflatexlater{2016/03/01}{
	    \newcommand{\wordboundary}{4095}}{
	    \newcommand{\wordboundary}{255}}
    \makeatother

    \newif\ifcode
    \codefalse
    \definecolor{Grey}{rgb}{0.40,0.40,0.40}
    %If using XeLaTeX, use magic to not highlight . operators with purple.
    \ifdefined\XeTeXcharclass
    \XeTeXinterchartokenstate = 1
    \newXeTeXintercharclass \mycharclassGrey
    \XeTeXcharclass `. \mycharclassGrey
    \XeTeXinterchartoks 0 \mycharclassGrey   = {\bgroup\ifcode\color{Grey}\else\fi}

    \XeTeXinterchartoks \wordboundary \mycharclassGrey = {\bgroup\ifcode\color{Grey}\else\fi}

    \XeTeXinterchartoks \mycharclassGrey 0   = {\egroup}
    \XeTeXinterchartoks \mycharclassGrey \wordboundary = {\egroup}
    \fi %end magical operator highlighting
    %End Reconfigured Pygments
    
   
    % Exact colors from NB
    \definecolor{incolor}{HTML}{303F9F}
    \definecolor{outcolor}{HTML}{D84315}
    \definecolor{cellborder}{HTML}{CFCFCF}
    \definecolor{cellbackground}{HTML}{F7F7F7}

    % needed definitions
    \newif\ifleftmargins
    \newlength{\promptlength}

    % cell style settings
        \leftmarginsfalse

    
    % prompt
    \newcommand{\prompt}[3]{
        \needspace{1.1cm}
        \settowidth{\promptlength}{ #1 [#3] }
        \ifleftmargins\hspace{-\promptlength}\hspace{-5pt}\fi
        {\color{#2}#1 [#3]:}
        \ifleftmargins\vspace{-2.7ex}\fi
    }
    
    
    % environments
    \newenvironment{OutVerbatim}{\VerbatimEnvironment%
        \begin{tcolorbox}[breakable, boxrule=.5pt, size=fbox, pad at break*=1mm, opacityfill=0]
            \begin{Verbatim}
            }{
            \end{Verbatim}
        \end{tcolorbox}
    }
    
    %Updated MathJax Compatibility (if future behaviour of the notebook changes this may be removed)
    \renewcommand{\TeX}{\ifmmode \textrm{\Oldtex} \else \textbackslash TeX \fi}
    \renewcommand{\LaTeX}{\ifmmode \Oldlatex \else \textbackslash LaTeX \fi}
    
    % Header Adjustments
    \renewcommand{\paragraph}{\textbf}
    \renewcommand{\subparagraph}[1]{\textit{\textbf{#1}}}

    
    % Prevent overflowing lines due to hard-to-break entities
    \sloppy 
    % Setup hyperref package
    \hypersetup{
      breaklinks=true,  % so long urls are correctly broken across lines
      colorlinks=true,
      urlcolor=urlcolor,
      linkcolor=linkcolor,
      citecolor=citecolor,
      }
    % Slightly bigger margins than the latex defaults
    \geometry{verbose,tmargin=.5in,bmargin=.7in,lmargin=.5in,rmargin=.5in}
    

    \begin{document}
    
    
    
    
    

    
    \hypertarget{assignment-3}{%
\section{Assignment 3}\label{assignment-3}}

    
\prompt{In}{incolor}{1}
\codetrue
\begin{tcolorbox}[breakable, size=fbox, boxrule=1pt, pad at break*=1mm, colback=cellbackground, colframe=cellborder]
\begin{minted}[breaklines=True]{ipython3}
import numpy as np 
import matplotlib as mp
import matplotlib.pyplot as plt

from numpy import sin, pi
from scipy.optimize import curve_fit 
from textwrap import wrap

%matplotlib inline
%config InlineBackend.figure_format = 'pdf'
\end{minted}
\end{tcolorbox}
\codefalse

    \hypertarget{question-1-polynomial-fit}{%
\subsection{Question 1: Polynomial
fit}\label{question-1-polynomial-fit}}

    We want to find a fourth degree polynomial fit that goes through the
points \((-2, 0)\), \((-1, -1)\), \((0, 0)\), \((1, 1)\), and
\((2, 0)\). To do this, we set up a system of linear equations:

\[\vec{y} = X \vec{a}\]

\[\text{Where} \left\{ \begin{array}{ll}
            \vec{y}  & \text{is the vector of the "y" values} \\
            \vec{X}  & \text{is the matrix of the "x" values} \\
            \vec{a}  & \text{is the vector of the coefficients of the polynomial}
            \end{array} \right.\]

Our polynomial will have the form:

\[y = a_0 + a_1 x + a_2 x^2 + a_3 x^3 + a_4 x^4\]

Substituting the given values into the equation gives us:

\[\begin{bmatrix}
    0\\
    -1\\
    0\\
    1\\
    0
\end{bmatrix}
=
\begin{bmatrix}
    1 & -2 & 4 & -8  & 16 \\
    1 & -1 & 1 & -1 & 1 \\
    1 & 0 & 0 & 0 & 0 \\
    1 & 1 & 1 & 1 & 1 \\
    1 & 2 & 4 & 8  & 16 
\end{bmatrix}
\begin{bmatrix}
    a_0 \\
    a_1 \\
    a_2 \\
    a_3 \\
    a_4
\end{bmatrix}\]

    To find \(\vec{a}\) we first find the inverse of \(X\), then
\(\vec{a} = X^{-1} \vec{y}\)

    
\prompt{In}{incolor}{2}
\codetrue
\begin{tcolorbox}[breakable, size=fbox, boxrule=1pt, pad at break*=1mm, colback=cellbackground, colframe=cellborder]
\begin{minted}[breaklines=True]{ipython3}
P = np.array([[-2, 0], [-1, -1], [0, 0], [1, 1], [2,0]])
x_space = np.arange(-2, 2, 0.01)
\end{minted}
\end{tcolorbox}
\codefalse

    
\prompt{In}{incolor}{3}
\codetrue
\begin{tcolorbox}[breakable, size=fbox, boxrule=1pt, pad at break*=1mm, colback=cellbackground, colframe=cellborder]
\begin{minted}[breaklines=True]{ipython3}
# Generate matrix "X"
def X(p=P):
    X = np.zeros((p.shape[0], p.shape[0]))
    for i in range(0, p.shape[0]):
        X[i] += p[:,0] ** i
    return X.T

# Invert "X"
def inv_X(p=P):
    inv_x = np.linalg.inv(X(p))
    return inv_x

# Find vector "a"
def vec_a(p=P):
    a =  inv_X(p) @ p[:,1]
    return a
\end{minted}
\end{tcolorbox}
\codefalse

    
\prompt{In}{incolor}{4}
\codetrue
\begin{tcolorbox}[breakable, size=fbox, boxrule=1pt, pad at break*=1mm, colback=cellbackground, colframe=cellborder]
\begin{minted}[breaklines=True]{ipython3}
# Get our vector "a"
a = vec_a()

# Equation for "y"
def y(X, A=a):
    y = 0
    for i in range(0, np.size(A)):
        y += a[i] * X**i
    return y
\end{minted}
\end{tcolorbox}
\codefalse

    
\prompt{In}{incolor}{5}
\codetrue
\begin{tcolorbox}[breakable, size=fbox, boxrule=1pt, pad at break*=1mm, colback=cellbackground, colframe=cellborder]
\begin{minted}[breaklines=True]{ipython3}
# Plot it out: 
# ============

fig, ax = plt.subplots(1, 1, figsize=(10, 6))

ax.scatter(P[:,0], P[:,1], c='black', label="Data points", zorder=5)
ax.plot(x_space, sin(pi*x_space / 2), label=r"$y=\sin \left(\frac{2 x}{\pi}\right)$")
ax.plot(x_space, y(x_space), label=r"$y$ = %5.3f + %5.3f $x$ + %5.3f $x^2$ + %5.3f $x^3$ + %5.3f $x^4$" % tuple(a))
ax.set_ylabel(r"$y$", rotation=0, fontsize=14)
ax.set_xlabel(r"$x$", fontsize=14)
ax.set_title(r"Polynomial fit", fontsize=16)

plt.legend()
plt.show()
\end{minted}
\end{tcolorbox}
\codefalse

    \begin{center}
    \adjustimage{max size={0.9\linewidth}{0.9\paperheight}}{output_8_0.pdf}
    \end{center}
    { \hspace*{\fill} \\}
    
    \hypertarget{question-2-cubic-spline-fit}{%
\subsection{Question 2: Cubic spline
fit}\label{question-2-cubic-spline-fit}}

    We want to find a cubic spline \(S(x)\) with boundary conditions
\(S'(-2)=S'(2)\) and \(S''(-2)=S''(2)\).

As we have five points, we need to find four equations for the functions
between these points. This gives us the equation of the cubic spline:

\[\left\{ \begin{array}{ll}
            S_0(x_0) = a_0 + b_0 x_0 + c_0 x_0^2 + d_0 x_0^3 \\
            S_0(x_1) = a_0 + b_0 x_1 + c_0 x_1^2 + d_0 x_1^3 \\
            S_1(x_1) = a_1 + b_1 x_1 + c_1 x_1^2 + d_1 x_1^3 \\
            S_1(x_2) = a_1 + b_1 x_2 + c_1 x_2^2 + d_1 x_2^3 \\
            S_2(x_2) = a_2 + b_2 x_2 + c_2 x_2^2 + d_2 x_2^3 \\
            S_2(x_3) = a_2 + b_2 x_3 + c_2 x_3^2 + d_2 x_3^3 \\
            S_3(x_3) = a_3 + b_3 x_3 + c_3 x_3^2 + d_3 x_3^3
            \end{array} \right.\]

    We want the spline to have continuous first derivatives:

\[\left\{ \begin{array}{ll}
            S'_0(x_1) = b_0 + 2 c_0 x_1 + 3 d_0 x_1^2 = b_1 + 2 c_1 x_1 + 3 d_1 x_1^2 = S'_1(x_1)\\
            S'_1(x_2) = b_1 + 2 c_1 x_2 + 3 d_1 x_2^2 = b_2 + 2 c_2 x_2 + 3 d_2 x_2^2 = S'_2(x_2)\\
            S'_2(x_3) = b_2 + 2 c_2 x_3 + 3 d_2 x_3^2 = b_3 + 2 c_3 x_3 + 3 d_3 x_3^2 = S'_3(x_3)
            \end{array} \right.\]

The boundary conditions require that:

\[S'_0(x_0) = b_0 + 2 c_0 x_0 + 3 d_0 x_0^2 = b_3 + 2 c_3 x_4 + 3 d_3 x_4^2 = S'_3(x_4)\]

    For a natural spline, we want the second order derivatives to be
continuous and to be zero at the ends:

\[\left\{ \begin{array}{ll}
            S''_0(x_1) = 2 c_0 + 6 d_0 x_1 = 2 c_1 + 6 d_1 x_1 = S''_1(x_1)\\
            S''_1(x_2) = 2 c_1 + 6 d_1 x_2 = 2 c_2 + 6 d_2 x_2 = S''_2(x_2)\\
            S''_2(x_3) = 2 c_2 + 6 d_2 x_3 = 2 c_3 + 6 d_3 x_3 = S''_3(x_3)\\
            S_0''(x_0) = 2 c_0 + 6 d_0 x_0 = 2 c_3 + 6 d_3 x_4 = S_3''(x_4) = 0
            \end{array} \right.\]

    Substituting the values into our equations, we have 13 equations to
solve for 12 unknowns:

\[\left\{ \begin{array}{ll}
            S_0(-2) = 0 = a_0 - 2 b_0 + 4 c_0 - 8 d_0  \\
            S_0(-1) = -1 = a_0 - b_0 + c_0 - d_0 \\
            S_1(-1) = -1 = a_1 - b_1 + c_1 - d_1 \\
            S_1(0) = 0 =a_1 \\
            S_2(0) = 0 =a_2 \\
            S_2(1) = 1 =  a_2 + b_2 + c_2 + d_2 \\
            S_3(1) = 1 =  a_3 + b_3 + c_3 + d_3 \\
            S_3(2) = 0 = a_3 + 2 b_3 + 4 c_3 + 8 d_3 \\
            S'_0(-1) = S'_1(-1) = b_0 - 2 c_0 + 3 d_0 = b_1 - 2 c_1 + 3 d_1 \\
            S'_1(0) = S'_2(0) = b_1 = b_2\\
            S'_2(1) = S'_3(1) = b_2 + 2 c_2 + 3 d_2 = b_3 + 2 c_3 + 3 d_3 \\
            S'_0(-2) = S'_3(2) = b_0 - 4 c_0 + 12 d_0 = b_3 + 4 c_3 + 12 d_3 \\
            S''_0(-1) = S''_1(-1) = 2 c_0 - 6 d_0 = 2 c_1 - 6 d_1 \\
            S''_1(0) = S''_2(0) = 2 c_1 = 2 c_2 \\
            S''_2(1) = S''_3(1) = 2 c_2 + 6 d_2 = 2 c_3 + 6 d_3 \\
            S_0''(-2) = S_3''(2) = 2 c_0 - 12 d_0 = 2 c_3 + 12 d_3 = 0
            \end{array} \right.\]

    We can see straight away that \(a_1 = a_2 = 0\),
\(c_0 - 6 d_0 = c_3 + 6 d_3 = 0\), and \(c_2 + 3 d_2 = c_3 + 3 d_3\),
therefore, substitute these into the expressions:

\[\left\{ \begin{array}{ll}
            a_0 - 2 b_0 + 4 c_0 - 8 d_0 = 0\\
            a_0 - b_0 + c_0 - d_0 = -1\\
            - b_1 + c_1 - d_1 = -1\\
            a_1 = 0\\
            a_2 = 0\\
            b_2 + c_2 + d_2 = 1\\
            a_3 + b_3 + c_3 + d_3 = 1\\
            a_3 + 2 b_3 + 4 c_3 + 8 d_3 = 0\\
            b_0 - 2 c_0 + 3 d_0 = b_1 - 2 c_1 + 3 d_1 \\
            b_1 = b_2\\
            b_2 + c_2 = b_3 + c_3 \\
            b_0 - 2 c_0 = b_3 + 2 c_3 \\
            c_0 - 3 d_0 = c_1 - 3 d_1 \\
            c_1 = c_2 \\
            c_2 + 3 d_2 = c_3 + 3 d_3 \\
            c_0 - 6 d_0 = c_3 + 6 d_3 = 0
            \end{array} \right.\]

    Now, let's put this into a matrix:

\[C \vec{a} = \vec{b}\]

\[\text{Where:} \ 
\left\{ \begin{array}{ll}
            C \ \text{is the matrix containing the coefficient of "a's", "b's", "c's", and "d's"}\\
            \vec{a} \ \text{is a vector containing the "a's", "b's", "c's", and "d's"}\\
            \vec{b} \ \text{is a vector containing the number on the right side of the equations}
            \end{array} \right.\]

    The coefficients can be calculated by: \[\vec{a} = C^{-1} \vec{b}\]

    
\prompt{In}{incolor}{6}
\codetrue
\begin{tcolorbox}[breakable, size=fbox, boxrule=1pt, pad at break*=1mm, colback=cellbackground, colframe=cellborder]
\begin{minted}[breaklines=True]{ipython3}
# Create an array "C"
C = np.array([
    [1, -2,  4, -8,  0,  0,  0,  0,  0,  0,  0,  0,  0,  0,  0,  0],
    [1, -1,  1, -1,  0,  0,  0,  0,  0,  0,  0,  0,  0,  0,  0,  0],
    [0,  0,  0,  0,  0, -1,  1, -1,  0,  0,  0,  0,  0,  0,  0,  0],
    [0,  0,  0,  0,  0,  0,  0,  0,  0,  1,  1,  1,  0,  0,  0,  0], 
    [0,  0,  0,  0,  0,  0,  0,  0,  0,  0,  0,  0,  1,  1,  1,  1], 
    [0,  0,  0,  0,  0,  0,  0,  0,  0,  0,  0,  0,  1,  2,  4,  8],  
    [0,  0,  0,  0,  0,  1,  0,  0,  0, -1,  0,  0,  0,  0,  0,  0], 
    [0,  0,  0,  0,  0,  0,  1,  0,  0,  0, -1,  0,  0,  0,  0,  0],  
    [0,  0,  0,  0,  0,  0,  0,  0,  0,  1,  1,  0,  0, -1, -1,  0], 
    [0,  1, -2,  0,  0,  0,  0,  0,  0,  0,  0,  0,  0, -1, -2,  0],
    [0,  0,  1, -3,  0,  0, -1,  3,  0,  0,  0,  0,  0,  0,  0,  0], 
    [0,  0,  0,  0,  0,  0,  0,  0,  0,  0,  1,  3,  0,  0, -1, -3], 
    [0,  1, -2,  3,  0, -1,  2, -3,  0,  0,  0,  0,  0,  0,  0,  0], 
    [0,  0,  1, -6,  0,  0,  0,  0,  0,  0,  0,  0,  0,  0, -1, -6], 
    [0,  0,  0,  0,  1,  0,  0,  0,  0,  0,  0,  0,  0,  0,  0,  0], 
    [0,  0,  0,  0,  0,  0,  0,  0,  1,  0,  0,  0,  0,  0,  0,  0]])

# Create an array of vector "b"
vec_b = np.array([
    [0],
    [-1],
    [-1],
    [1], 
    [1], 
    [0],  
    [0], 
    [0],  
    [0], 
    [0],
    [0], 
    [0], 
    [0], 
    [0], 
    [0], 
    [0]])
\end{minted}
\end{tcolorbox}
\codefalse

    
\prompt{In}{incolor}{7}
\codetrue
\begin{tcolorbox}[breakable, size=fbox, boxrule=1pt, pad at break*=1mm, colback=cellbackground, colframe=cellborder]
\begin{minted}[breaklines=True]{ipython3}
# Find the inverse of "C"
C_inv = np.linalg.inv(C)

# Take the product of "C" and "b"
vec_a = C_inv @ vec_b
print(r"a = ", vec_a)
\end{minted}
\end{tcolorbox}
\codefalse

    \begin{Verbatim}[commandchars=\\\{\}]
a =  [[ 1.00000000e+00]
 [ 4.50000000e+00]
 [ 3.00000000e+00]
 [ 5.00000000e-01]
 [ 0.00000000e+00]
 [ 1.50000000e+00]
 [ 2.22044605e-16]
 [-5.00000000e-01]
 [ 0.00000000e+00]
 [ 1.50000000e+00]
 [ 2.22044605e-16]
 [-5.00000000e-01]
 [-1.00000000e+00]
 [ 4.50000000e+00]
 [-3.00000000e+00]
 [ 5.00000000e-01]]

    \end{Verbatim}

    This is our coefficients for the cubic spline equations.

    \hypertarget{a-b-equations-and-derivatives-of-the-cubic-spline-equations}{%
\subsubsection{a) \& b): Equations and derivatives of the cubic spline
equations}\label{a-b-equations-and-derivatives-of-the-cubic-spline-equations}}

Cubic spline equations:

\[\left\{ \begin{array}{ll}
            S_0(x) = 1 + 4.5 x + 3 x^2 + 0.5 x^3 \\
            S_1(x) = 1.5 x - 0.5 x^3 \\
            S_2(x) = 1.5 x - 0.5 x^3 \\
            S_3(x) = -1 + 4.5 x - 3 x^2 + 0.5 x^3
            \end{array} \right.\]

    Derivatives of the cubic spline equations:

\[\left\{ \begin{array}{ll}
            S_0(x) = 4.5 + 6 x + 1.5 x^2 \\
            S_1(x) = 1.5 - 1.5 x^2 \\
            S_2(x) = 1.5 - 1.5 x^2 \\
            S_3(x) = 4.5 - 6 x + 1.5 x^2
            \end{array} \right.\]

    \hypertarget{c-plot-it-out}{%
\subsubsection{c) Plot it out:}\label{c-plot-it-out}}

    
\prompt{In}{incolor}{8}
\codetrue
\begin{tcolorbox}[breakable, size=fbox, boxrule=1pt, pad at break*=1mm, colback=cellbackground, colframe=cellborder]
\begin{minted}[breaklines=True]{ipython3}
x_space0 = np.arange(-2, -1, 0.01)
x_space1 = np.arange(-1, 0, 0.01)
x_space2 = np.arange(0, 1, 0.01)
x_space3 = np.arange(1, 2, 0.01)

def S(A=vec_a, x0=x_space0, x1=x_space1, x2=x_space2, x3=x_space3):
    s0 = A[0] + A[1]*x0 + A[2]*x0**2 + A[3]*x0**3
    s1 = A[4] + A[5]*x1 + A[6]*x1**2 + A[7]*x1**3
    s2 = A[8] + A[9]*x2 + A[10]*x2**2 + A[11]*x2**3
    s3 = A[12] + A[13]*x3 + A[14]*x3**2 + A[15]*x3**3
    return s0, s1, s2, s3
\end{minted}
\end{tcolorbox}
\codefalse

    
\prompt{In}{incolor}{40}
\codetrue
\begin{tcolorbox}[breakable, size=fbox, boxrule=1pt, pad at break*=1mm, colback=cellbackground, colframe=cellborder]
\begin{minted}[breaklines=True]{ipython3}
# Plot it out:
# ============
S0, S1, S2, S3 = S()

fig, ax2 = plt.subplots(1, 1, figsize=(10, 6))

ax2.scatter(P[:,0], P[:,1], c='black', label="Data points", zorder=5)
ax2.plot(x_space0, S0, label=r"$S_0(x) = {0:.2f} + {1:.2f} x + {2} x^2 + {3} x^3$"
         .format(vec_a[0,0], vec_a[1,0], vec_a[2,0], vec_a[3,0]), linewidth=2)
ax2.plot(x_space1, S1, label=r"$S_1(x) = {0} + {1} x + {2:.2f} x^2 + {3:.2f} x^3$"
         .format(vec_a[4,0], vec_a[5,0], 0, vec_a[7,0]), linewidth=2)
ax2.plot(x_space2, S2, label=r"$S_2(x) = {0} + {1} x + {2:.2f} x^2 + {3:.2f} x^3$"
         .format(vec_a[8,0], vec_a[9,0], 0, vec_a[11,0]), linewidth=2)
ax2.plot(x_space3, S3, label=r"$S_3(x) = {0:.2f} + {1:.2f} x + {2:.2f} x^2 + {3:.2f} x^3$"
         .format(vec_a[12,0], vec_a[13,0], vec_a[14,0], vec_a[15,0]), linewidth=2)
ax2.plot(x_space, sin(pi*x_space / 2), label=r"$y=\sin \left(\frac{2 x}{\pi}\right)$", linewidth=4, zorder=0)

ax2.set_ylabel(r"$y$", rotation=0, fontsize=14)
ax2.set_xlabel(r"$x$", fontsize=14)
ax2.set_title(r"Cubic spline fit", fontsize=16)

plt.ylim(-1.75, 1.25)
plt.legend()
plt.show()
\end{minted}
\end{tcolorbox}
\codefalse

    \begin{center}
    \adjustimage{max size={0.9\linewidth}{0.9\paperheight}}{output_24_0.pdf}
    \end{center}
    { \hspace*{\fill} \\}
    
    \hypertarget{question-3}{%
\subsection{Question 3}\label{question-3}}

    \hypertarget{a-simulate-the-paths}{%
\subsubsection{a) Simulate the paths}\label{a-simulate-the-paths}}

    
\prompt{In}{incolor}{10}
\codetrue
\begin{tcolorbox}[breakable, size=fbox, boxrule=1pt, pad at break*=1mm, colback=cellbackground, colframe=cellborder]
\begin{minted}[breaklines=True]{ipython3}
paths = 100
time = 50

# Change in steps for random walk
RW = np.random.randint(0, 2, size=(paths, time+1)) * 2 - 1
RW[:, 0] = 0

# Position of the random walkers
RW_pos = np.cumsum(RW, axis=1)

# Time
t = np.arange(0, np.shape(RW_pos)[1], 1)
\end{minted}
\end{tcolorbox}
\codefalse

    
\prompt{In}{incolor}{11}
\codetrue
\begin{tcolorbox}[breakable, size=fbox, boxrule=1pt, pad at break*=1mm, colback=cellbackground, colframe=cellborder]
\begin{minted}[breaklines=True]{ipython3}
fig, ax3 = plt.subplots(1, 1, figsize=(10, 4))

for i in range(0, paths):
    plt.plot(t, RW_pos[i, :])
    i += 1

plt.title(r"Positions for {0} random walkers for a time of {1}".format(paths, time), fontsize=16)    
plt.xlabel(r"Time ($n$)", fontsize=14)
plt.ylabel(r"Position ($x$)", fontsize=14)
plt.show()
\end{minted}
\end{tcolorbox}
\codefalse

    \begin{center}
    \adjustimage{max size={0.9\linewidth}{0.9\paperheight}}{output_28_0.pdf}
    \end{center}
    { \hspace*{\fill} \\}
    
    \hypertarget{b-plot-the-variance-for-each-time-step}{%
\subsubsection{b) Plot the variance for each time
step}\label{b-plot-the-variance-for-each-time-step}}

    
\prompt{In}{incolor}{12}
\codetrue
\begin{tcolorbox}[breakable, size=fbox, boxrule=1pt, pad at break*=1mm, colback=cellbackground, colframe=cellborder]
\begin{minted}[breaklines=True]{ipython3}
# Find the variance
var = np.nanvar(RW_pos, axis=0)
\end{minted}
\end{tcolorbox}
\codefalse

    
\prompt{In}{incolor}{13}
\codetrue
\begin{tcolorbox}[breakable, size=fbox, boxrule=1pt, pad at break*=1mm, colback=cellbackground, colframe=cellborder]
\begin{minted}[breaklines=True]{ipython3}
fig, ax4 = plt.subplots(1, 1, figsize=(10, 4))

ax4.plot(t, var**0.5)

plt.title(r"Standard deviations at each time step for {0} random walkers for a time of {1}".format(paths, time), fontsize=16)    
plt.xlabel(r"Time ($n$)", fontsize=14)
plt.ylabel(r"Standard deviation ($\sigma$)", fontsize=14)
plt.show()
\end{minted}
\end{tcolorbox}
\codefalse

    \begin{center}
    \adjustimage{max size={0.9\linewidth}{0.9\paperheight}}{output_31_0.pdf}
    \end{center}
    { \hspace*{\fill} \\}
    
    \hypertarget{c-show-sigma-anr}{%
\subsubsection{\texorpdfstring{c) Show:
\(\sigma = an^r\)}{c) Show: \textbackslash{}sigma = an\^{}r}}\label{c-show-sigma-anr}}

    
\prompt{In}{incolor}{14}
\codetrue
\begin{tcolorbox}[breakable, size=fbox, boxrule=1pt, pad at break*=1mm, colback=cellbackground, colframe=cellborder]
\begin{minted}[breaklines=True]{ipython3}
# Define function
# ===============
def sdev_fit(t, a, r):
    f = a * t**r
    return f
\end{minted}
\end{tcolorbox}
\codefalse

    
\prompt{In}{incolor}{15}
\codetrue
\begin{tcolorbox}[breakable, size=fbox, boxrule=1pt, pad at break*=1mm, colback=cellbackground, colframe=cellborder]
\begin{minted}[breaklines=True]{ipython3}
# Fit the curve
# =============

popt, pcov = curve_fit(sdev_fit, t, var**0.5)
\end{minted}
\end{tcolorbox}
\codefalse

    
\prompt{In}{incolor}{16}
\codetrue
\begin{tcolorbox}[breakable, size=fbox, boxrule=1pt, pad at break*=1mm, colback=cellbackground, colframe=cellborder]
\begin{minted}[breaklines=True]{ipython3}
fig, ax5 = plt.subplots(1, 1, figsize=(10, 4))

ax5.plot(t, var**0.5)
ax5.plot(t, sdev_fit(t, *popt), '-', label=r'Fit ($\sigma = a n^r$): $a$=%5.3f, $r$=%5.3f' % tuple(popt))

plt.title(r"Standard deviations at each time step for {0} random walkers for a time of {1}".format(paths, time), fontsize=16)    
plt.xlabel(r"Time", fontsize=14)
plt.ylabel(r"Standard deviation ($\sigma$)", fontsize=14)
plt.legend()
plt.show()
\end{minted}
\end{tcolorbox}
\codefalse

    \begin{center}
    \adjustimage{max size={0.9\linewidth}{0.9\paperheight}}{output_35_0.pdf}
    \end{center}
    { \hspace*{\fill} \\}
    
    As we can see, the simulation suggests the standard deviation follows:

\[\sigma = an^r\]

\[\text{Where:} 
  \left\{ \begin{array}{ll}
      a \approx 0.952 \\
      r \approx 0.522
  \end{array} \right.\]


    % Add a bibliography block to the postdoc
    
    
    
    \end{document}
